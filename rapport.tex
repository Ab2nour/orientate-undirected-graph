\documentclass{article}      % Specifies the document class

% -------------------- Packages --------------------
\usepackage{amsmath}
\usepackage{amssymb}
\usepackage[noend]{algpseudocode}
\usepackage{algorithm}
\usepackage{graphicx}
\usepackage{float}
\usepackage{fontawesome5}
\usepackage{listings}

\usepackage{hyperref}
\hypersetup{
    colorlinks=true,
    linkcolor=blue,
    filecolor=magenta,      
    urlcolor=cyan,
    pdftitle={Overleaf Example},
    pdfpagemode=FullScreen,
    }

\urlstyle{same}

\usepackage{bookmark}
\hypersetup{hidelinks} %enlève les cadres rouges autour des hyperliens


% ---------- PSEUDO CODE : hack to remove indent ----------
% https://tex.stackexchange.com/questions/354564/how-to-remove-leading-indentation-from-algorithm
\usepackage{xpatch}
\makeatletter
\xpatchcmd{\algorithmic}
  {\ALG@tlm\z@}{\leftmargin\z@\ALG@tlm\z@}
  {}{}
\makeatother

\usepackage{xcolor}
\usepackage[framemethod=tikz]{mdframed}
\usepackage{tikzpagenodes}
\usetikzlibrary{calc}


% -------------------- Couleurs --------------------
\definecolor{definition}{HTML}{2f80ed}
\definecolor{definition-bg}{HTML}{e0ecfd}
\definecolor{exogris}{gray}{0.4}

% -------------------- Styles --------------------
\mdfdefinestyle{mystyle}{%
  innertopmargin=10px,
  innerbottommargin=10px,
  linecolor=definition,
  backgroundcolor=definition-bg,
  roundcorner=4px
}
\newmdenv[style=mystyle]{definition}


% -------------------- Document --------------------
\title{Théorie des Graphes\\\Large{Projet noté}}
\author{MADANI Abdenour\\TRIOLET Hugo}
\date{Licence 3\\2021 - 2022}
\begin{document}
\normalsize
\maketitle

\renewcommand*\contentsname{Table des matières}
\tableofcontents
\newpage

\section{Introduction}
Les objectifs de ce TPs sont :
\begin{itemize}
  \item Orienter un graphe non-orienté en un graphe fortement connexe,
  \item Décomposer en graphe en chaînes (selon Schmidt$^\text{\cite{schmidt}}$),
  \item Établir si un graphe est 2-connexe ou 2-arête,
  \item Calculer les composantes 2-connexes et 2-arêtes-connexes sinon.
\end{itemize}

On utilisera pour ceci \textbf{SageMath} (bibliothèque de fonctions pour Python).

\section{Définitions}





\begin{definition}
{ \scriptsize \textcolor{definition}{\faIcon{graduation-cap} \textbf{DÉFINITION}}}
\vspace{3px}
\\ \underline{\textbf{Pont}}
\vspace{2.5px}
\\ Arête dont la suppression augmente le nombre de composantes connexes du graphe restant.%
\\ \textit{(aussi appelé arête déconnectante)}
\end{definition}

\begin{definition}
{ \scriptsize \textcolor{definition}{\faIcon{graduation-cap} \textbf{DÉFINITION}}}
\vspace{3px}
\\ \underline{\textbf{Sommet d'articulation}}
\vspace{2.5px}
\\ Sommet dont la suppression augmente le nombre de composantes connexes du graphe restant.%
\\ \textit{(aussi appelé sommet déconnectant ou noeud d'articulation)}
\end{definition}

\begin{definition}
{ \scriptsize \textcolor{definition}{\faIcon{graduation-cap} \textbf{DÉFINITION}}}
\vspace{3px}
\\ \underline{\textbf{2-connexité (\textit{2-sommet-connexité})}}
\vspace{2.5px}
\\ Un graphe est dit 2-connexe s'il n'admet pas de sommet d'articulation.
\end{definition}

\begin{definition}
{ \scriptsize \textcolor{definition}{\faIcon{graduation-cap} \textbf{DÉFINITION}}}
\vspace{3px}
\\ \underline{\textbf{2-arête-connexité}}
\vspace{2.5px}
\\ Un graphe est dit 2-arête-connexe s'il n'admet pas de pont.
\end{definition}

\begin{definition}
{ \scriptsize \textcolor{definition}{\faIcon{graduation-cap} \textbf{DÉFINITION}}}
\vspace{3px}
\\ \underline{\textbf{Depth-First Search (DFS)}}
\vspace{2.5px}
\\ Parcours en profondeur d'un graphe.
\end{definition}

\begin{definition}
{ \scriptsize \textcolor{definition}{\faIcon{graduation-cap} \textbf{DÉFINITION}}}
\vspace{3px}
\\ \underline{\textbf{Depth-First Index (DFI)$^\text{\cite{schmidt}}$}}
\vspace{2.5px}
\\ Date à laquelle le DFS a \textbf{débuté} sur un noeud.
\end{definition}

\section{Résumé de notre approche}
On implémente la décomposition en chaînes$^\text{\cite{schmidt}}$.
\\Celle-ci nous permet d'obtenir tous les ponts, ainsi que tous les sommets d'articulation du graphe.
\\On en déduit ensuite si le graphe est 2-connexe, 2-arête-connexe, ou aucun des deux, grâce à l'Algorithme 1 de Schmidt$^\text{\cite{schmidt}}$.
\\On calcule ensuite les composantes 2-connexes et 2-arêtes-connexes.
%

\section{Exercice 1}
\textcolor{exogris}{
Dans une premier temps il est demandé d’implémenter les algorithmes de calcul de 2-connexité dû à Schmidt$^\text{\cite{schmidt}}$.
\\(1) Calculer les composantes 2-connexes d’un graphe.
\\(2) Calculer les composantes 2-arêtes-connexes d’un graphe.
}
\\A

\section{Exercice 2}
\textcolor{exogris}{
Trouvez un orientation en un graphe fortement connexe$^\text{\cite{schmidt}}$ ou trouvez une arête déconnectante.
}
\\On effectue la décomposition en chaînes telle que décrite par Schmidt.
\\Celle-ci va orienter toutes les arêtes (cf \textit{Figure 1b}$^\text{\cite{schmidt}}$).
\\D'après le théorème de Robbins$^\text{\cite{Robbins}}$, 

\section{todo}
todo
\\\\Pour obtenir les composantes 2-arêtes-connexes :
\\On prend le graphe original, on supprime ses ponts.
\\Ensuite, on enlève les noeuds restants de degré 0.
%
\\\\Pour obtenir les composantes 2-connexes :
\\On prend le graphe original, 
\\Ensuite, on enlève les noeuds restants de degré 0.
\section{Exemples de graphes testés}

\section{Références}
\begin{thebibliography}{9}
\bibitem{robbins}
Herbert Robbins, \textit{A theorem on graphs, with an application to a problem on traffic control}, American Mathematical Monthly 46, 281–283.

\bibitem{schmidt}
Jens M. Schmidt, \href{https://arxiv.org/ftp/arxiv/papers/1209/1209.0700.pdf}{\underline{\textit{A Simple Test on 2-Vertex- and 2-Edge-Connectivity}}},
\\Inf. Process. Lett. \textbf{113} (2013), no. 7, 241–244.
\end{thebibliography}

\end{document}
