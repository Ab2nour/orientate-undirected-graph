\documentclass{article}      % Specifies the document class

\usepackage{amsmath}
\usepackage{amssymb}
\usepackage[noend]{algpseudocode}
\usepackage{algorithm}
\usepackage{graphicx}
\usepackage{float}
\usepackage{fontawesome5}

\usepackage{hyperref}
\hypersetup{
    colorlinks=true,
    linkcolor=blue,
    filecolor=magenta,      
    urlcolor=cyan,
    pdftitle={Overleaf Example},
    pdfpagemode=FullScreen,
    }

\urlstyle{same}

\usepackage{bookmark}
\hypersetup{hidelinks} %enlève les cadres rouges autour des hyperliens


% ---------- hack to remove indent ----------
% https://tex.stackexchange.com/questions/354564/how-to-remove-leading-indentation-from-algorithm
\usepackage{xpatch}
\makeatletter
\xpatchcmd{\algorithmic}
  {\ALG@tlm\z@}{\leftmargin\z@\ALG@tlm\z@}
  {}{}
\makeatother


\title{Théorie des Graphes\\\Large{Projet noté}}
\author{MADANI Abdenour\\TRIOLET Hugo}
\date{Licence 3\\2021 - 2022}
\begin{document}
\normalsize
\maketitle

\renewcommand*\contentsname{Table des matières}
\tableofcontents
\newpage

\section{Introduction}
Les objectifs de ce TPs sont :
\begin{itemize}
  \item Orienter un graphe non-orienté en un graphe fortement connexe,
  \item Décomposer en graphe en chaînes (selon Schmidt$^\text{\cite{schmidt}}$),
  \item Établir si un graphe est 2-connexe ou 2-arête,
  \item Calculer les composantes 2-connexes et 2-arêtes-connexes sinon.
\end{itemize}

On utilisera pour ceci \textbf{SageMath} (bibliothèque de fonctions pour Python).


\section{Exemples de graphes testés}

\section{Références}
\begin{thebibliography}{9}
\bibitem{schmidt}
Jens M. Schmidt
\\\href{https://arxiv.org/ftp/arxiv/papers/1209/1209.0700.pdf}{\underline{\textit{A Simple Test on 2-Vertex- and 2-Edge-Connectivity}}}, (2013).
\end{thebibliography}

\end{document}
